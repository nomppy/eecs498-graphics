\documentclass{article}
\title{EECS 498 - Homework 2 report}
\author{Kenneth Sun}

\usepackage{listings}
\usepackage{color}
\usepackage{graphicx}

\definecolor{dkgreen}{rgb}{0,0.6,0}
\definecolor{gray}{rgb}{0.5,0.5,0.5}
\definecolor{mauve}{rgb}{0.58,0,0.82}

\lstset{frame=tb,
  language=Java,
  aboveskip=3mm,
  belowskip=3mm,
  showstringspaces=false,
  columns=flexible,
  basicstyle={\small\ttfamily},
  numbers=none,
  numberstyle=\tiny\color{gray},
  keywordstyle=\color{blue},
  commentstyle=\color{dkgreen},
  stringstyle=\color{mauve},
  breaklines=true,
  breakatwhitespace=true,
  tabsize=3
}

\lstset{language=C++}

\begin{document}
\maketitle

\section{Diffuse-only trace}
\begin{lstlisting}
Vec3 Scene::trace(const Ray &ray, int bouncesLeft, bool discardEmission) {
    if constexpr(DEBUG) {
        assert (ray.isNormalized());
    }
    if (bouncesLeft < 0) return {};

    Intersection inter = getIntersection(ray);
    if (!inter.happened) return {};

    return inter.getDiffuseColor();
}
\end{lstlisting}

\section{Direct illumination w/o acceleration}
\begin{lstlisting}
Vec3 Scene::trace(const Ray &ray, int bouncesLeft, bool discardEmission) {
    if constexpr(DEBUG) {
        assert (ray.isNormalized());
    }
    if (bouncesLeft < 0) return {};

    Intersection inter = getIntersection(ray);
    if (!inter.happened) return {};

    Ray omega_o { inter.pos, Random::randomHemisphereDirection(inter.getNormal()) };
    Intersection light_inter { getIntersection(omega_o) };
    if (light_inter.happened) {
        Vec3 brdf { inter.calcBRDF(-omega_o.dir, -ray.dir)};
        float cosine_term = omega_o.dir.dot(inter.getNormal());
        return inter.getEmission() + (2*PI * cosine_term) * brdf * light_inter.getEmission();
    }

    return {};
}
\end{lstlisting}

Output with SPP=32:\\
\includegraphics[width=0.5\textwidth]{DI-noaccel.png}

\textbf{Thought question: Why 16 SPP looks darker than 128 SPP?} This makes sense because unsampled points will be dark
and 16SPP has more unsampled points.

\section{Global illumination w/o acceleration}
\begin{lstlisting}
Vec3 Scene::trace(const Ray &ray, int bouncesLeft, bool discardEmission) {
    if constexpr(DEBUG) {
        assert (ray.isNormalized());
    }
    if (bouncesLeft == 0) return {};

    Intersection inter = getIntersection(ray);
    if (!inter.happened) return {};

    Ray omega_o { inter.pos, Random::randomHemisphereDirection(inter.getNormal()) };
    Intersection light_inter { getIntersection(omega_o) };
    if (light_inter.happened) {
        Vec3 brdf { inter.calcBRDF(-omega_o.dir, -ray.dir)};
        float cosine_term = omega_o.dir.dot(inter.getNormal());
        return inter.getEmission() + (2*PI * cosine_term) * brdf * trace(omega_o, bouncesLeft-1, false);
    }

    return {};
}
\end{lstlisting}
Output with SPP=32:\\
\includegraphics[width=0.5\textwidth]{GI-noaccel.png}

\section{Global illumination w/ acceleration}
\begin{lstlisting}
Vec3 Scene::trace(const Ray &ray, int bouncesLeft, bool discardEmission) {
    if constexpr(DEBUG) {
        assert (ray.isNormalized());
    }
    if (bouncesLeft == 0) return {};

    Intersection inter = getIntersection(ray);
    if (!inter.happened) return {};

    Vec3 direct_rad {};
    Intersection sample { sampleLight() };
    Vec3 light_dir { sample.pos - inter.pos };
    float dlight { light_dir.getLength() };
    Ray ray_to_light { inter.pos, light_dir / dlight };
    if (getIntersection(ray_to_light).object == sample.object) {
        Vec3 brdf { inter.calcBRDF(ray.dir, ray_to_light.dir) };
        float cosine_term = inter.getNormal().dot(ray_to_light.dir);
        float light_cosine_term = sample.getNormal().dot(-ray_to_light.dir);
        direct_rad = sample.getEmission() * (lightArea * cosine_term * light_cosine_term * brdf / (dlight * dlight));
    }

    Ray omega_i { inter.pos, Random::cosWeightedHemisphere(inter.getNormal()) };
    Intersection omega_inter { getIntersection(omega_i) };
    if (omega_inter.happened) {
        Vec3 brdf { inter.calcBRDF(-omega_i.dir, -ray.dir)};
        if (discardEmission)
            return direct_rad + (PI * brdf) * trace(omega_i, bouncesLeft-1, true);
        else
            return inter.getEmission() + direct_rad + (PI * brdf) * trace(omega_i, bouncesLeft-1, true);
    }

    return {};
}
\end{lstlisting}
\includegraphics[width=0.5\textwidth]{GI-accel.png}
\end{document}}
